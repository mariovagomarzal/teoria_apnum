\chapter[Aproximación de funciones]
  {Aproximación de funciones. El problema de aproximación de Taylor}

\section{Aproximación de funciones}
  Uno de los objetivos principales en Aproximación numérica es aproximar
  una función $f$ por otra $f^*$ escogida de un conjunto de funciones
  $\mathcal{C}$ que consideramos más \textit{sencillas}.

  Si la función $f$ se ha de someter a un proceso numérico $\mathcal{P}$,
  como por ejemplo hallar ceros de la función, calcular una integral, etc.,
  diremos que $\mathcal{P}(f^*)$ es una \textit{aproximación} de
  $\mathcal{P}(f)$.

  Está claro que un problema de aproximación depende de la clase
  $\mathcal{C}$ donde se buscan las funciones y de la regla o norma que se
  considere para \textit{imitar el compartaiento} de $f$ por $f^*$.

  En estos apuntes trataremos principalmente problemas lineales. Es decir,
  la función $f^*$ se construye como combinación lineal de unas funciones
  $\{\phi_i\}_{i=0}^n \subseteq \mathcal{C}$ prefijadas. Así pues, el
  objetivo es determinar los coeficientes $c_0, c_1, \dots, c_n$ de manera
  que
  \[
    f^* = \sum_{i=0}^n c_i \phi_i
    = c_0 \phi_0 + c_1 \phi_1 + \dots + c_n \phi_n.
  \]

  En la mayor parte de estos apuntes usaremos \textit{polinomios
  aproximadores}, esto es, el espacio de funciones que tomaremos será
  \[
    \mathcal{C} = \Pi_n := 
    \set{a_0 + a_1 x + \dots + a_n x^n :
    a_i \in \RR, \forall 0 \le i \le n}.
  \]

\section[El problema de Taylor]{El problema de aproximación de Taylor}
  Si consideramos que \textit{imitar el comportamiento} de $f$ por $f^*$
  significa que $f^*$ y $f$ y sus respectivas derivadas hasta un cierto
  orden coinciden en un punto, entonces, el problema de aproximación de
  Taylor es el problema de aproximación más natural que podemos plantear.
  Estudiemos dicho problema desde la perspectiva de la aproximación
  numérica.

  \begin{definition}[Problema de aproximación de Taylor]
    Sea $f$ una función escalar $n$ veces derivable en un intervalo $[a,
    b]$ y sea $x_0 \in (a, b)$. El problema de aproximación de Taylor trata
    de determinar un polinomio de grado menor o igual que $n$, 
    $P_n \in \Pi_n$, tal que
    \[
      P_n^{(k)} = f^{(k)}(x_0), \quad \forall 0 \le k \le n.
    \]
  \end{definition}

  \begin{theorem}
    El problema de aproximación de Taylor tiene soluión y es única. 
  \end{theorem}

  \begin{proof}
    Para determinar el polinomio $P_n$ que resuelve el problema de
    aproximación de Taylor, consideraremos la base canónica
    $\set{x^i}_{i=0}^{n}$ de $\Pi_n$, de manera que el objetivo será
    determinar los coeficientes $a_0, a_1, \dots, a_n$ de manera que
    \[
      P_n(x) = a_0 + a_1 x + \dots + a_n x^n
    \]
    cumpla los requisitos del problema.

    Imponiendo las condiciones descritas en el problema de aproximación de
    Taylor, obtenemos el siguiente sistema de ecuaciones lineales
    \[
      \underbrace{
        \begin{pmatrix}
          1 & x_0 & x_0^2 & \cdots & x_0^n \\
          0 & 1 & 2 x_0 & \cdots & n x_0^{n-1} \\
          \vdots & \vdots & \vdots & \ddots & \vdots \\
          0 & 0 & 0 & \cdots & n! \\
        \end{pmatrix}
      }_A
      \underbrace{
        \begin{pmatrix}
          a_0 \\ a_1 \\ \vdots \\ a_n
        \end{pmatrix}
      }_X
      =
      \underbrace{
        \begin{pmatrix}
          f(x_0) \\ f'(x_0) \\ \vdots \\ f^{(n)}(x_0)
        \end{pmatrix}
      }_F.
    \]

    Como $A$ es una matriz triangular superior con elementos diagonales no
    nulos, tenemos que $\det A \neq 0$ y, por tanto, el sistema tiene
    solución única.
  \end{proof}

  El método utilizado para demostrar la existencia no es el convencional
  para \textit{construir} el polinomio de Taylor\footnote{Ya podemos hablar
  del \guillemot{polinomio de Taylor} puesto que hemos probado su
  unicidad.}. Es costumbre presentar el polinomio de Taylor como
  \begin{equation} \label{eq:taylor_natural}
    \emphmath{
      P_n(x) = \sum_{k=0}^n \frac{f^{(k)}(x_0)}{k!} (x - x_0)^k.
    }
  \end{equation}

  Es fácil comprobar que el polinomio anterior verifica las condiciones del
  problema y, por la unicidad ya probada, es el mismo polinomio que hemos
  obtenido en la demostración anterior.

  La forma de \ref{eq:taylor_natural} es la que se conoce como
  \textit{forma natural} del polinomio de Taylor, en la que usamos la base
  \[
    \set{1, (x - x_0), \dots, (x - x_0)^n}
  \]
  de $\Pi_n$. Usando esta base, el cálculo de los coeficientes en la
  expresión
  \[
    P_n(x) = c_0 + c_1 (x - x_0) + \dots + c_n (x - x_0)^n
  \]
  es trivial:
  \[
    c_k = \frac{f^{(k)}(x_0)}{k!}, \quad \forall 0 \le k \le n.
  \]
  
  Destacamos, por último, que el polinomio de Taylor en forma natural posee
  la propiedad de \textit{permanencia}, esto es, si queremos calcular el
  polinomio de Taylor de orden $n + 1$ de una función $f$ a partir del
  polinomio de Taylor de orden $n$, basta con añadir un término más a la
  suma de \ref{eq:taylor_natural}.

  \subsection{El error del polinomio de Taylor}
    Es importante conocer el error que se comete al aproximar una función
    $f$ por su polinomio de Taylor. Los siguientes resultados abordan este
    problema.

    \begin{theorem}[Error de Lagrange para el polinomio de Taylor]
      \label{thm:error_taylor}
      Sea $f$ una función escalar $n + 1$ veces derivable en $[a, b]$ y
      $x_0 \in (a, b)$. Si $P_n(x) \in \Pi_n$ es el polinomio de Taylor en
      el punto $x_0$, entonces, para cada $x \in (a, b)$, existe un $\xi_x
      \in (a, b)$ tal que
      \begin{equation} \label{eq:error_lagrange}
        \emphmath{
          f(x) - P_n(x) = \frac{f^{(n+1)}(\xi_x)}{(n+1)!}(x - x_0)^{n+1}.
        }
      \end{equation}
    \end{theorem}

    \begin{proof}
      En primer lugar, notamos que si $x = x_0$, ambos miembros de
      \ref{eq:error_lagrange} se anulan. Así, supongamos que $x \neq x_0$.
      Para este $x$, definimos
      \[
        M := \frac{f(x) - P_n(x)}{(x - x_0)^{n+1}}.
      \]
      Probaremos que existe un $\xi_x \in (a, b)$ tal que
      \[
        M = \frac{f^{(n+1)}(\xi_x)}{(n+1)!},
      \]
      lo que probará el teorema.

      Consideramos la función auxiliar
      \[
        \phi(t) = P_n(t) + M (t - x_0)^{n+1} - f(t).
      \]
      Por construcción, notamos que $\phi$ es derivable en $(x, x_0)$ (o
      $(x_0, x)$) y que $\phi(x) = \phi(x_0) = 0$. De esta manera, por el
      teorema de Rolle, sabemos que $\phi'$ se anula en algún punto $t_1$
      entre $x$ y $x_0$. De nuevo, por el teorema de Rolle, $\phi''$ se
      anula en algún punto $t_2$ entre $t_1$ y $x_0$. Repitiendo este
      proceso $n + 1$ veces, obtenemos que $\phi^{(n+1)}$ se anula en algún
      punto $\xi_x$ entre $t_n$ y $x_0$. En particular, $\xi_x$ está entre
      $x$ y $x_0$ y, por tanto, en $(a, b)$.

      Notamos que
      \[
        \phi^{(n+1)}(t) = M (n+1)! - f^{(n+1)}(t)
      \]
      y que, por la elección de $\xi_x$, $\phi^{(n+1)}(\xi_x) = 0$. De
      esta manera, obtenemos que
      \[
        M = \frac{f^{(n+1)}(\xi_x)}{(n+1)!},
      \]
      como queríamos probar.
    \end{proof}

    \begin{remark} \label{rmk:error_taylor}
      Por simplicidad, en el teorema \ref{thm:error_taylor} hemos dicho que
      $\xi_x$ está en $(a, b)$. Sin embargo, como hemos visto en la
      demostración, podemos ser más precisos y decir que $\xi_x$ está en el
      intervalo abierto $(x, x_0)$ o $(x_0, x)$, dependiendo de si $x <
      x_0$ o $x > x_0$.

      En algunas ocasiones, puede ser útil tomar el intervalo más
      pequeño posible para obtener mejores cotas del error.
    \end{remark}

    \subsubsection{Acotaciones del error}
      En el teorema \ref{thm:error_taylor} hemos probado la existencia de
      un punto $\xi_x$ que verifica \ref{eq:error_lagrange}. Sin embargo,
      este resultado no es constructivo, es decir, no nos permite
      determinar el valor exacto de $\xi_x$. Por ello, lo máximo a lo que
      podemos aspirar, en general, es a acotar el error cometido al
      aproximar $f(x)$ por $P_n(x)$.

      En la práctica, esto se traduce en acotar el valor absoluto de la
      derivada $(n+1)$-ésima de $f$ en el intervalo $(a, b)$ o, como
      comentábamos en la nota \ref{rmk:error_taylor}, en el intervalo $(x,
      x_0)$.
      
      Es decir, si llamamos
      \[
        M = \sup_{t \in (a, b)} \abs{f^{(n+1)}(t)},
      \]
      entonces, por \ref{eq:error_lagrange}, tenemos que
      \[
        \abs{f(x) - P_n(x)}
        \le \frac{M}{(n+1)!} \abs{x - x_0}^{n+1}.
      \]

  \subsection{Caracterizaciones del polinomio de Taylor}
    En esta sección, veremos algunas caracterizaciones del polinomio de
    Taylor que nos permitirán comprobar si un posible candidato a ser dicho
    polinomio lo es realmente sin necesidad de calcular explícitamente las
    derivadas de $f$ u otras operaciones que puedan ser costosas.

    \begin{proposition} \label{prop:caracterizacion1_taylor}
      Sea $f$ una función escalar $n + 1$ veces derivable en $[a, b]$ y
      $x_0 \in (a, b)$. El polinomio de Taylor de $f$ en $x_0$ de orden $n$
      es el único polinomio $P_n \in \Pi_n$ tal que
      \begin{equation} \label{eq:caracterizacion1_taylor}
        \lim_{x \to x_0} \frac{P_n(x) - f(x)}{(x - x_0)^n} = 0.
      \end{equation}
    \end{proposition}

    \begin{proof}
      Primeramente, supongamos que $P_n$ verifica
      \ref{eq:caracterizacion1_taylor}. En este caso, es claro que
      $f(x_0) = P_n(x_0)$, pues en caso contrario, el límite no sería
      finito.

      Por otra parte, como
      \[
        \lim_{x \to x_0} f(x) - P_n(x) = 0
        \quad \text{y} \quad
        \lim_{x \to x_0} (x - x_0)^n = 0
      \]
      y ambas funciones son derivables, podemos aplicar la regla de
      L'Hôpital para obtener que
      \[
        \lim_{x \to x_0} \frac{f'(x) - P_n'(x)}{n (x - x_0)^{n-1}} = 0,
      \]
      pues, si el límite no fuese finito, el límite de la función original
      tampoco lo sería y tendríamos una contradicción. De esta manera,
      usando un razonamiento análogo al anterior, obtenemos que $f'(x_0) =
      P_n'(x_0)$.

      No es difícil ver que, usando un razonamiento inductivo similar,
      tenemos que
      \[
        f^{(k)}(x_0) = P_n^{(k)}(x_0), \quad \forall 0 \le k \le n,
      \]
      lo que prueba que $P_n$ es el polinomio de Taylor de $f$ en $x_0$.

      Por otra parte, supongamos que $P_n$ es el polinomio de Taylor de $f$
      en $x_0$. Por el teorema \ref{thm:error_taylor}, sabemos que existe
      un $\xi_x \in (a, b)$ tal que
      \[
        f(x) - P_n(x) = \frac{f^{(n+1)}(\xi_x)}{(n+1)!}(x - x_0)^{n+1}.
      \]

      Sustituyendo en el límite de \ref{eq:caracterizacion1_taylor},
      tenemos que
      \[
        \begin{split}
          \lim_{x \to x_0} \frac{P_n(x) - f(x)}{(x - x_0)^n}
          &= \lim_{x \to x_0} \frac{f^{(n+1)}(\xi_x)}{(n+1)!}(x - x_0) \\
          &= \frac{f^{(n+1)}(x_0)}{(n+1)!} \lim_{x \to x_0} (x - x_0) \\
          &= 0,
        \end{split}
      \]
      como queríamos probar.
    \end{proof}

    \begin{proposition}
      Sea $f$ una función escalar $n + 1$ veces derivable en $[a, b]$ y
      $x_0 \in (a, b)$. El polinomio de Taylor de $f$ en $x_0$ de orden $n$
      es el único polinomio $P_n \in \Pi_n$ tal que
      \begin{equation} \label{eq:caracterizacion2_taylor}
        \abs{f(x) - P_n(x)} \le M \abs{x - x_0}^{n+1},
      \end{equation}
      para cierto $M > 0$, para todo $x$ en algún entorno de $x_0$.
    \end{proposition}

    \begin{proof}
      Supongamos primeramente que $P_n$ es el polinomio de Taylor de $f$ en
      $x_0$. Por el teorema \ref{thm:error_taylor}, sabemos que existe un
      $\xi_x \in (a, b)$ tal que
      \[
        \lim_{x \to x_0} \frac{P_n(x) - f(x)}{(x - x_0)^{n + 1}}
        = \lim_{x \to x_0} \frac{f^{(n+1)}(\xi_x)}{(n+1)!} =: \alpha.
      \]

      Así, por la definición de límite, si tomamos $\varepsilon = 1$,
      existe un $\delta > 0$ tal que
      \[
        \abs{\frac{P_n(x) - f(x)}{(x - x_0)^{n + 1}} - \alpha} < 1,
        \quad \forall x \in (x_0 - \delta, x_0 + \delta).
      \]

      Por la desigualdad triangular, tenemos que
      \[
        \abs{\frac{f(x) - P_n(x)}{(x - x_0)^{n + 1}}}
        < \abs{\alpha} + 1 =: M, 
        \quad \forall x \in (x_0 - \delta, x_0 + \delta),
      \]
      o, equivalentemente,
      \[
        \abs{f(x) - P_n(x)} < M \abs{x - x_0}^{n + 1},
        \quad \forall x \in (x_0 - \delta, x_0 + \delta),
      \]
      como queríamos probar.

      Por otra parte, supongamos que $P_n$ verifica
      \ref{eq:caracterizacion2_taylor} en cierto entorno de $x_0$. En este
      caso,
      \[
        \abs{\frac{f(x) - P_n(x)}{(x - x_0)^n}} \le M \abs{x - x_0},
      \]
      para todo $x$ en dicho entorno y con $x \neq x_0$.

      Por tanto, tomando límites, tenemos que
      \[
        \lim_{x \to x_0} \frac{f(x) - P_n(x)}{(x - x_0)^n} = 0,
      \]
      lo que, por la proposición \ref{prop:caracterizacion1_taylor}, prueba
      que $P_n$ es el polinomio de Taylor de $f$ en $x_0$.
    \end{proof}
